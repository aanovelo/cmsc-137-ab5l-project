%% LyX 2.0.2 created this file.  For more info, see http://www.lyx.org/.
%% Do not edit unless you really know what you are doing.
\documentclass[english]{article}
\usepackage[T1]{fontenc}
\usepackage[latin9]{inputenc}

\makeatletter
%%%%%%%%%%%%%%%%%%%%%%%%%%%%%% User specified LaTeX commands.
\usepackage{draftwatermark}
\SetWatermarkLightness{0.92}
\SetWatermarkScale{5}

\makeatother

\usepackage{babel}
\begin{document}

\title{SciCloud: Cloud for high-performance scientific computing}


\date{Joseph Anthony C. Hermocilla}

\maketitle

\section*{Thesis}

High-performance scientific computing can be done over cloud infrastructure
and can be guaranteed a performance that is comparable to bare metal
machines currently used in supercomputers or dedicated commodity clusters.


\section*{Rationale}

Since the digital computer was invented, it has been an important
tool in doing science. In fact, without it, progress in science will
be quite slow. Several scientific breakthroughs have been made possible
because of digital computers. From the sequencing of the human genome
to the discovery of the Higg's Boson particle, the digital computer
has its very important role. From simple arithmetic to modeling biological
and physical systems, the digital computer is able to deliver the
result of experiments in an accurate and timely manner.

Scientific applications, such as equation solvers and simulations,
are very CPU-intensive and Memory-intensive. These applications, because
of their heavy computational and storage requirements, are usually
run in high-performance computing (HPC) infrastructures using supercomputers
or commodity clusters for distributed computations. Setting up such
infrastructures requires a lot of initial capital and technical investments.
Once the infrastructure has been set up, maintenance costs will also
be required to update the hardware and software components. Although
these infrastructures are indeed useful, studies have shown that they
are underutilized\cite{ostermann_performance_2010}. Their usage depends
on the computing demand pattern of scientists and researchers. Most
HPC clusters in research centers cater to a mix of researchers from
different study areas. For example, physicists and chemists may share
the same HPC cluster running on Linux operating system. Differences
in usage requirements of these researchers affect the overall operation
of the cluster. Provisioning the hardware and software requirements
needed for scientific computing is thus very challenging because of
these factors.

Despite the recent advances in hardware (increased processing speed
and memory) and their decreasing costs, typical users do not harness
the full capacity of their machines, either because the software they
are using does not support them or their computing needs do not require
access to the full capacity of their machines. Web browsing and word
processing, for example, usually do not require high-performance machines
whereas gaming needs them for an enhanced user experience. Another
example is the case of laboratory computers used for instruction in
academic institutions. Despite having the latest processors and a
good amount of memory, they are typically used just for simple programming
exercises which do not require the full access to hardware performance.
These underutilized machines can be used for scientific computing
needs. 

Machines used in clusters are usually dedicated, which means that
they cannot be used for other general-purpose computing needs. When
running tasks, the number of machines used from the clusters used
are usually statically set often with the assistance of system administrators
managing the cluster. A scientist who needs to perform a simulation
need to contact the cluster administrator to schedule the job that
must be run.


\subsection*{Nature of Scientific Applications}


\section*{Objectives}

The main objective of this research is to develop cloud computing
solutions to improve high-performance scientific computing. Specifically,
this work aims to
\begin{enumerate}
\item evaluate existing public cloud offerings (Amazon EC2, Microsoft Azure)
for applicability in scientific computing tasks;
\item deploy a private cloud setup using OpenStack for testing; 
\item identify and describte the characteristics of scientific applications
that make it a challenge to execute them on the cloud; and 
\item develop a novel framework and architecture to make it easy and efficient
for scientists to perform scientific experiments on the cloud.
\end{enumerate}

\section*{Methodology}

\bibliographystyle{plain}
\bibliography{proposal}

\end{document}
