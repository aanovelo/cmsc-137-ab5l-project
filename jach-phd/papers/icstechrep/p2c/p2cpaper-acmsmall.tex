% v2-acmsmall-sample.tex, dated March 6 2012
% This is a sample file for ACM small trim journals
%
% Compilation using 'acmsmall.cls' - version 1.3 (March 2012), Aptara Inc.
% (c) 2010 Association for Computing Machinery (ACM)
%
% Questions/Suggestions/Feedback should be addressed to => "acmtexsupport@aptaracorp.com".
% Users can also go through the FAQs available on the journal's submission webpage.
%
% Steps to compile: latex, bibtex, latex latex
%
% For tracking purposes => this is v1.3 - March 2012

\documentclass[prodmode,acmtecs]{acmsmall} % Aptara syntax

% Package to generate and customize Algorithm as per ACM style
\usepackage[ruled]{algorithm2e}
\renewcommand{\algorithmcfname}{ALGORITHM}
\SetAlFnt{\small}
\SetAlCapFnt{\small}
\SetAlCapNameFnt{\small}
\SetAlCapHSkip{0pt}
\IncMargin{-\parindent}

% Metadata Information
\acmVolume{2014}
\acmNumber{1}
\acmArticle{2}
\acmYear{2014}
\acmMonth{6}

% Document starts
\begin{document}

% Page heads
\markboth{J. A. C. Hermocilla}{Peak2Cloud: Scientific Computing on the Cloud}

% Title portion
\title{Peak2Cloud: Scientific Computing on the Cloud}
\author{JOSEPH ANTHONY C. HERMOCILLA
\affil{University of the Philippines Los Banos}}
% NOTE! Affiliations placed here should be for the institution where the
%       BULK of the research was done. If the author has gone to a new
%       institution, before publication, the (above) affiliation should NOT be changed.
%       The authors 'current' address may be given in the "Author's addresses:" block (below).
%       So for example, Mr. Abdelzaher, the bulk of the research was done at UIUC, and he is
%       currently affiliated with NASA.

\begin{abstract}
Peak2Cloud (P2C) is an Openstack-based private cloud for scientific and high performance computing.
First, we present how P2C was configured and tested. Then we describe vcluster, a tool for 
rapidly deploying message-passing clusters on P2C. Lastly, we analyze some benchmark results on the 
performance of P2C deployed virtual clusters.
\end{abstract}

%\category{C.2.2}{Computer-Communication Networks}{Network Protocols}

%\terms{Design, Algorithms, Performance}

%\keywords{Wireless sensor networks, media access control,
%multi-channel, radio interference, time synchronization}

%acmformat{Gang Zhou, Yafeng Wu, Ting Yan, Tian He, Chengdu Huang, John A. Stankovic,
%and Tarek F. Abdelzaher, 2010. A multifrequency MAC specially
%designed for  wireless sensor network applications.}
% At a minimum you need to supply the author names, year and a title.
% IMPORTANT:
% Full first names whenever they are known, surname last, followed by a period.
% In the case of two authors, 'and' is placed between them.
% In the case of three or more authors, the serial comma is used, that is, all author names
% except the last one but including the penultimate author's name are followed by a comma,
% and then 'and' is placed before the final author's name.
% If only first and middle initials are known, then each initial
% is followed by a period and they are separated by a space.
% The remaining information (journal title, volume, article number, date, etc.) is 'auto-generated'.

\begin{bottomstuff}
This work is supported by the Department of Science and Technology (DOST) Accelerated Science 
and Technology 
Human Resource Development Program (ASTHRDP). 

Author's addresses: J. A. C. Hermocilla, Institute of Computer Science,
University of the Philippines Los Banos
\end{bottomstuff}

\maketitle


\section{Introduction}
Cloud computing has become a buzzword in today's modern computing, though there is no agreed 
upon meaning of the term. In 2011, NIST \cite{mell_nist_2011} published a definition that is 
widely quoted and used. The popularity of cloud computing mainly comes from its ability to provision 
additional resources on demand with minimum intervention from the provider. It leverages advances 
in virtualization and web services technologies. For example, a website with a sudden increase 
in workload can start another server machine (virtual) almost instantaneously to accommodate 
the additional load.

Cloud computing offers service models which include Software-as-a-Service(SaaS), 
Platform-as-a-Service(PaaS), and Infrastructure-as-a-Service(IaaS). IaaS allows the consumer 
to provision computing resources(hardware, network, storage) to run arbitrary 
software including operating systems \cite{mell_nist_2011}. 


\section{Related Work}

\section{Methodology}

\section{Results and Discussion}

\section{Conclusions}


% Acknowledgments
\begin{acks}
The author would like to thank the Lord.
\end{acks}

% Bibliography
\bibliographystyle{ACM-Reference-Format-Journals}
\bibliography{cloudcomp}
%\bibliography{acmsmall-sample-bibfile}
                             % Sample .bib file with references that match those in
                             % the 'Specifications Document (V1.5)' as well containing
                             % 'legacy' bibs and bibs with 'alternate codings'.
                             % Gerry Murray - March 2012

% History dates
\received{May 2014}{June 2014}{June 2014}

\end{document}
% End of v2-acmsmall-sample.tex (March 2012) - Gerry Murray, ACM


