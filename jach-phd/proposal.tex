%% LyX 2.0.2 created this file.  For more info, see http://www.lyx.org/.
%% Do not edit unless you really know what you are doing.
\documentclass[english]{article}
\usepackage[T1]{fontenc}
\usepackage[latin9]{inputenc}

\makeatletter
%%%%%%%%%%%%%%%%%%%%%%%%%%%%%% User specified LaTeX commands.
%\usepackage{draftwatermark}
%\SetWatermarkLightness{0.92}
%\SetWatermarkScale{5}

\makeatother

\usepackage{babel}
\begin{document}

\title{SciCloud: Cloud for high-performance scientific computing}


\date{Joseph Anthony C. Hermocilla}

\maketitle

\section{Thesis}

High-performance scientific computing can be done over cloud infrastructure
and can be guaranteed a performance that is comparable to bare metal
machines currently used in supercomputers or dedicated commodity clusters.


\section{Rationale}

Since the digital computer was invented, it has been an important
tool in doing science. In fact, without it, progress in science will
be quite slow. Several scientific breakthroughs have been made possible
because of digital computers. From the sequencing of the human genome
to the discovery of the Higg's Boson particle, the digital computer
has its very important role. From simple arithmetic to modeling biological
and physical systems, the digital computer is able to deliver the
result of experiments in an accurate and timely manner.

Scientific applications, such as equation solvers and simulations,
are very CPU-intensive and Memory-intensive. These applications, because
of their heavy computational and storage requirements, are usually
run in high-performance computing (HPC) infrastructures using supercomputers
or commodity clusters for distributed computations. Setting up such
infrastructures requires a lot of initial capital and technical investments.
Once the infrastructure has been set up, maintenance costs will also
be required to update the hardware and software components. Although
these infrastructures are indeed useful, studies have shown that they
are underutilized. Their usage depends on the computing demand pattern
of scientists and researchers which often make these infrastructures
idle. Most HPC clusters in research centers cater to a mix of researchers
from different study areas. For example, physicists and chemists may
share the same HPC cluster running on Linux operating system. Differences
in usage requirements of these researchers affect the overall operation
of the cluster. Provisioning the hardware and software requirements
needed for scientific computing is thus very challenging because of
these factors.

Despite the recent advances in hardware (increased processing speed
and memory) and their decreasing costs, typical users do not harness
the full capacity of their machines, either because the software they
are using does not support them or their computing needs do not require
access to the full capacity of their machines. Web browsing and word
processing, for example, usually do not require high-performance machines
whereas gaming needs them for an enhanced user experience. Another
example is the case of laboratory computers used for instruction in
academic institutions. Despite having the latest processors and a
good amount of memory, they are typically used just for simple programming
exercises which do not require the full access to hardware performance.
These underutilized machines can be used for scientific computing
needs. 

Machines used in clusters are usually dedicated, which means that
they cannot be used for other general-purpose computing needs. When
running tasks, the number of machines used from the clusters used
are usually statically set often with the assistance of system administrators
managing the cluster. A scientist who needs to perform a simulation
need to contact the cluster administrator to schedule the job that
must be run.


\subsection{What is Scientific Computing?}

According to Wikipedia, Scientific Computing, also known as Computational
Science, involves the use of computers for solving problems in science.
It is concerned with model construction, which is mostly mathematical,
as well as quantitative analysis. It is also involved in computer
simulation and other forms of computation that are used to solve problems
in various scientific disciplines. Scientists develop software that
models the system being studied and run them with various parameters.
The nature of the models requires a lot computations(floating point
operations) and data.


\subsection{Desktop Computers, Supercomputers, Clusters, and Grid Computing}

The nature of software created for science, involving complex calculations,
will require a different approach to their implementation and execution.
Desktop computers are useful for small scale simulations but are slow
for complex experiments. Supercomputers are computers, way beyond
the capabilities of desktop computers, that have a specialized architecture
optimized for floating point operations. However, these types of computers
are usually very costly. Clusters are a collection of typical desktop
computers connected through a local area network and is less costly.
They execute scientific applications in a distributed manner usually
using distributed shared memory or message passing. Grid computing
extends the capabilities of clusters by going beyond the local area
network extending through a wide area network, in a totally distributed
manner. Grid computing is usually federated. 


\subsection{What is Cloud Computing?}

Cloud computing is a recent buzzword in computing. The diversity of
applications running on top of the Internet, from e-commerce and banking
to social networks, forces the vendors to address the issue of scale.
In the case of social networking sites, for example, the continuously
increasing number of users will demand additional physical computing
resources to be provisioned. Application developers who develop software
can focus on implementing functionality instead on optimizing code
to address increasing demand. 

The widely accepted definition of cloud computing is from the National
Institute of Standards and Technology\cite{mell_nist_2011}. It defines
cloud computing as 
\begin{quotation}
``a model for enabling ubiquitous, convenient, on-demand network
access to a shared pool of configurable computing resources (e.g.,
networks, servers, storage, applications, and services) that can be
rapidly provisioned and released with minimal management effort or
service provider interaction . This cloud model promotes availability
and is composed of five essential characteristics , three service
models , and four deployment models .''
\end{quotation}
There are five main characteristics of cloud computing\cite{mell_nist_2011}.
First, on-demand access allows users to provision computing resources
without requiring human interaction from the service provider. Second,
the capabilities are available over the network accessible through
standard mechanisms. Third, the resources are pooled and serve multiple
users in a multi-tenant model. The physical and virtual resources
are dynamically assigned and reassigned depending on the demand. Fourth,
provisioning of capabilities is elastic to quickly scale out and rapidly
released to quickly scale in. Lastly, cloud systems provide a metering
capability at some level of abstraction appropriate to the type of
service.

Service models for cloud computing include Software-as-a-Service(SaaS),
Platform-as-a-Service(PaaS), and Infrastructure-as-a-Service(IaaS)\cite{mell_nist_2011}.
SaaS allows users to access and use software running on a cloud infrastructure,
for example Google Docs. PaaS allows users to deploy their own applications
in a cloud infrastructure using programming languages and frameworks
supported by the provider. IaaS provides the user with the most basic
computing resources such as processor, storage, and network.

There are four ways to deploy a cloud\cite{mell_nist_2011}. A private
cloud is operated for a single organization which can be managed by
the organization itself or a third party and can be housed within
or outside the organization's premises. A community cloud is shared
by several organizations to support a specific community. Public clouds
are available for the general public. Lastly, hybrid clouds is a composition
of two or more clouds. 

Another notable characterization of cloud computing was developed
at the University of California Berkeley\cite{armbrust_view_2010}.


\section{Related Work}

A few studies have been made regarding the use of cloud for scientific
computing. In 2010, Ostermann et. al. conducted a performance analysis
of EC2 cloud computing services for scientific computing using microbenchmarks
and kernels. In 2011, Iosup et. al. conducted a performance analysis
of cloud computing services for Many-Tasks Scientific Computing\cite{iosup_performance_2011}.
In 2012, Zhao et. al. presented an early effort in designing and building
CloudDragon, a scientific cloud platform which is based on OpenNebula.
Most recently, Ludescher et. al. developed a cloud-based execution
frameworks for scientific problem solving environments which combines
a private and public cloud setup\cite{ludescher_cloud-based_2013}.In
2010, Schad et. al. performed runtime measurements in the cloud and
noted the variability due to the multitenant nature of the cloud\cite{schad_runtime_2010}.


\section{Objectives}

The main objective of this research is to develop cloud computing
solutions to improve high-performance scientific computing. Specifically,
this work aims to
\begin{enumerate}
\item evaluate existing public cloud offerings (Amazon EC2, Microsoft Azure)
for applicability in scientific computing tasks;
\item develop and deploy a private cloud over commodity desktop computers
which satisfy scientific applications requirements; 
\item develop a novel framework and platform to make it easy and efficient
for scientists to develop scientific applications; and
\item deploy a private cloud with software for scientific applications.
\end{enumerate}

\section{Methodology}


\paragraph{To achieve Objective 1}
\begin{enumerate}
\item Obtain accounts from Amazon EC2 and Microsoft Azure.
\item Review existing benchmarks and execute them on the platforms mentioned
above.
\end{enumerate}

\paragraph{To achieve Objective 2}
\begin{enumerate}
\item Identify key characteristics of scientific applications in terms of
computation, storage, and network.
\item Deploy a 5-node private cloud setup for testing.
\item Study internal architecture of existing cloud computing frameworks
(Eucalyptus\cite{nurmi_eucalyptus_2009} and OpenStack\cite{sefraoui_openstack:_2012})
to identify modules that can be modified to support the requirements
of scientific applications.
\item Design and implement a cloud system with an architecture and implementation
specific to support scientific applications.
\end{enumerate}

\paragraph{To achieve Objective 3}
\begin{enumerate}
\item Identify key characteristics of scientific applications in terms of
architectural and data representation.
\item Identify patterns in creating scientific applications and develop
software abstractions.
\item Create an application programming interface (API) for scientific applications.
\item Create sample applications using the newly created API.
\end{enumerate}

\paragraph{To achieve Objective 4}
\begin{enumerate}
\item Survey a group of scientists to identify their most commonly used
scientific applications.
\item Partially implement the features of the identified applications using
the API developed in Objective 3
\item Deploy the application as a Software-as-a-Service.
\end{enumerate}
\bibliographystyle{plain}
\bibliography{proposal}

\end{document}
